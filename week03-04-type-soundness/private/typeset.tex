% \usepackage[utf8]{inputenc}
\usepackage[left=2.5cm,top=1.5cm,right=2.5cm,bottom=2.5cm]{geometry}
\usepackage{graphicx}
\usepackage{enumerate}
\usepackage{xcolor}

\usepackage{amsmath,amssymb,amsfonts}
% \usepackage{mathrsfs}   % for \mathscr{.}
\usepackage{amsthm,thmtools,mathtools}
\usepackage{dashbox}

\usepackage{bussproofs}
\EnableBpAbbreviations
\usepackage{textgreek}

% [T1]{fontenc} and beramono for tt font of '<'
\usepackage[T1]{fontenc}
\usepackage{beramono}

\usepackage{pifont}
\usepackage{hyperref,nameref}

\newcommand{\deftechtag}[2]{\emph{\textbf{\hypertarget{#1}{#2}}}}
\newcommand{\techtag}[2]{\hyperlink{#1}{#2}}
\newcommand{\deftech}[1]{\deftech{#1}{#1}}
\newcommand{\tech}[1]{\techtag{#1}{#1}}

\declaretheoremstyle[
    spaceabove=3pt,
    spacebelow=3pt,
    headfont=\scshape\bfseries,
    bodyfont=\normalfont,
    headpunct={.},
    postheadspace={ },
    headindent={},
    notefont={\fontseries\mddefault\upshape}
]{scheadstyle}
\declaretheorem[parent=section,style=scheadstyle]{theorem}
\declaretheorem[sibling=theorem,style=scheadstyle]{lemma}
\declaretheorem[name=Proposition,sibling=theorem,style=scheadstyle]{prop}
\declaretheorem[name=Property,sibling=theorem,style=scheadstyle]{property}
\declaretheorem[name=Definition,sibling=theorem,style=scheadstyle]{defn}
\declaretheorem[sibling=theorem,style=scheadstyle]{corollary}
\declaretheorem[style=remark,name=Note,numbered=no,style=scheadstyle]{sidenote}
\declaretheorem[style=remark,numbered=no,style=scheadstyle]{claim}
\declaretheorem[style=remark,numbered=no,style=scheadstyle]{remark}

\newenvironment{mathl}{\begin{array}{l}}{\end{array}}
\newenvironment{mathllequ}{\left\{\begin{array}{ll}}{\end{array}\right.}

\newcommand{\icase}{\item \textbf{Case}\,}

\newcommand{\miff}{\text{ iff }}
\newcommand{\mwhere}{\text{ where }}
\newcommand{\mif}{\text{ if }}
\newcommand{\mand}{\text{ and }}
\newcommand{\mor}{\text{ or }}
\newcommand{\mfor}{\text{ for }}

\newcommand{\inn}{\notin}
\newcommand{\Ga}{\alpha}
\newcommand{\Gb}{\beta}
\newcommand{\tGS}{\textnormal{\textSigma}}
\newcommand{\tGs}{\textnormal{\textrho}}
\newcommand{\GS}{\Sigma}
\newcommand{\Gs}{\sigma}
\newcommand{\GL}{\Lambda}
\newcommand{\Gl}{\lambda}
\newcommand{\tGl}{\text{\textlambda}}
\newcommand{\Gr}{\rho}
\newcommand{\tGr}{\text{\textrho}}
\newcommand{\GD}{\Delta}
\newcommand{\Gd}{\delta}
\newcommand{\Gp}{\pi}
\newcommand{\tGp}{\textnormal{\textpi}}
\newcommand{\GG}{\Gamma}
\newcommand{\Gg}{\gamma}
\newcommand{\Gt}{\tau}
\newcommand{\Gth}{\theta}
\newcommand{\GTh}{\Theta}
\newcommand{\Gk}{\kappa}

\newcommand{\bbB}{\mathbb{B}}
\newcommand{\bbN}{\mathbb{N}}
\newcommand{\bbZ}{\mathbb{Z}}
\newcommand{\bbQ}{\mathbb{Q}}
\newcommand{\bbR}{\mathbb{R}}
\newcommand{\bbC}{\mathbb{C}}

\newcommand{\st}[1]{\{ #1 \}}

\newcommand{\FV}{\mathit{FV}\!}
\newcommand{\BV}{\mathit{BV}\!}
\newcommand{\dom}{\text{dom}}
\newcommand{\im}{\text{Im}}

\newcommand{\rul}[1]{\textsc{[#1]}}

\newcommand{\THole}{\mathnormal{[]}}
\newcommand{\inhole}[2]{#1\!\left[ #2 \right]}
\newcommand{\subst}[3]{\assubst{#1}{\dosubst{#2}{#3}}}
\newcommand{\substt}[5]{\assubst{#1}{\dosubst{#2,#4}{#3,#5}}}
\newcommand{\assubst}[2]{#1\!\left[{#2}\right]}
\newcommand{\dosubst}[2]{\left.\!{#2} \,\right/ #1}
\newcommand{\fresh}[3]{\text{fresh } #1 \notin #2(#3)}
\newcommand{\update}[3]{{#1\!\left[#2 \mapsto #3\right]}}

\DeclareMathOperator{\mapstoD}{\overset{\smash{\scriptscriptstyle\mathit{d}}\vphantom{x}}{\mapsto}}
\DeclareMathOperator{\tteq}{\textnormal{\texttt{=}}}

\newcommand{\tup}[1]{\left\langle {#1} \right\rangle}
